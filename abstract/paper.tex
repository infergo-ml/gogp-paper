\documentclass[sigplan,review]{acmart}\settopmatter{printfolios=true,printccs=false,printacmref=false}
\usepackage{url}
\usepackage{hyperref}

\acmConference[LAFI'2019]{Languages for Inference}{January 15,
2019}{Lisbon, Portugal}

\setcopyright{none}

\bibliographystyle{ACM-Reference-Format}


\title{Server-Side Probabilistic Programming}
\author{David Tolpin}
\affiliation{
    \institution{PUB+}
    \country{Israel}
}
\email{david.tolpin@gmail.com}

\begin{abstract}
	We share our experiences from integrating probabilistic
	programming into a server-side software system and
	present our ongoing work on probabilistic programming
	facility for Go, a modern programming language of choice for
	server-side software development. We demonstrate how a
	lightweight efficient probabilistic programming facility can
	be added to an existing programming language.  Server-side
	application of probabilistic programming poses challenges
	for a probabilistic programming system. We discuss the
	challenges and our experience in overcoming them, and
	suggest guidelines which can help in a wider adoption of
	probabilistic programming in server-side software systems.
\end{abstract}


\begin{document}
\maketitle

\section{Challenges of Server-Side Probabilistic Programming}

Incorporating a probabilistic program, or rather a probabilistic
procedure, within a larger code body appears to be rather
straightforward: one implements the model in the probabilistic
programming language, fetches and preprocesses the data in the
host programming language, passes the data and the model to an
inference algorithm, and post-processes the results in the
host programming language again to make algorithmic
decisions based on inference outcomes. However, complex
server-side software systems make integration of probabilistic
inference challenging. 

\paragraph{Simulation-inference dichotomy} Probabilistic models
often follow the design pattern of simulation-inference: a
significant part of the model is a simulator, running an
algorithm with fixed parameters; the optimal parameters, or
their distribution, are to be inferred. The inferred parameters
are then used by the software system to execute the simulation
independently of inference for forecasting and decision making.

This pattern suggest re-use of the simulator: instead of
implementing the simulator twice, in the probabilistic model and
in the host system, the same code can serve both purposes.
However to achieve this, the host language must coincide with
the implementation language of the probabilistic model, on one
hand, and allow a computationally efficient implementation of
the simulation, on the other hand. Some probabilistic systems
(Figaro~\cite{P09}, Anglican~\cite{TMY+16}, Turing~\cite{GXG18})
are built with tight integration with the host environment in
mind; more often than not though the probabilistic code is
not trivial to re-use.

\paragraph{Data interface} In a server-side application data
for inference comes from a variety of sources: network,
databases, distributed file systems, and in many different
formats. Efficient inference depends on fast data access and
updating. Libraries for data access and manipulation are
available in the host environment. While the host environment
can be used as a proxy retrieving and transforming the data,
such as in the case of Stan~\cite{Stan17} integrations,
sometimes direct access from the probabilistic code is the
preferred option, for example when the data is streamed or
retrieved conditionally. 

\paragraph{Integration and deployment} Deployment of server-side
software systems is a delicate process involving automatic
builds and maintenance of dependencies. Adding  a component
which possibly introduces additional software dependencies or
even a separate runtime complicates deployment. Minimizing the
burden of probabilistic programming on the integration
and deployment process should be a major consideration in
design or selection of probabilistic programming tools.
Probabilistic programming systems which are implemented or 
provide an interface in a popular programming language, e.g.
Python (Edward~\cite{THS+17}, Pyro~\cite{Pyro18}) are easier
to integrate and deploy, however the smaller the footprint
of a probabilistic system, the easier is the adoption.

\section{Probabilistic Programming Facility for Go}

Based on the experience of developing and deploying solutions
using different probabilistic environments, we came up with
guidelines to implementation of a probabilistic programming
facility for server-side applications. We believe that these
guidelines, when followed, help easier integration of
probabilistic programming inference into large-scale server-side
software systems.

\begin{enumerate}
\item A probabilistic model should be programmed in the host
programming language. The facility may impose a discipline on
model implementation, such as through interface constraints, but
otherwise supporting unrestricted use of the host language for
implementation of the model.

\item Built-in and user-defined data structures and libraries
should be accessible in the probabilistic programming model.
Inference techniques relying on the code structure, such as
those based on automatic differentiation, should support the
use of common data structures of the host language.

\item The model code should be reusable between inference and
simulation. The code which is not required solely for inference
should be written once for both inference of parameters and use
of the parameters in the host environment.  It should be
possible to run simulation outside the probabilistic model without
runtime or memory overhead imposed by inference needs.
\end{enumerate}

In line with the guidelines, we have implemented a probabilistic
programming facility for the Go programming language,
\texttt{infergo} (\url{http://infergo.ml/}). We have chosen Go
because Go is a small but expressive programming language with
efficient implementation, which has recently become quite
popular for computation-intensive server-side programming. At
the time of writing, the implementation is still at an early
stage. However, this facility is already used in production
environment for inference of mission-critical algorithm
parameters.  

A probabilistic model in \texttt{infergo} is an implementation
of the \texttt{Model} interface requiring a single method
\texttt{Observe} which accepts a vector (a Go \textit{slice}) of
floats, the parameters to infer, and returns a single float,
interpreted as unnormalized log-likelihood of the posterior
distribution. The implementation of the methods can be written
in virtually unrestricted Go and use any Go libraries.

For inference, \texttt{infergo} relies on automatic
differentiation. The source code of the model is
translated by a command-line tool provided by \texttt{infergo}
into an equivalent model with reverse-mode automatic
differentiation of the log-likelihood with respect 
to the parameters applied. The differentiation operates
on the built-in floating-point type and incurs only a small
computational overhead. However, even this overhead is avoided
when the model code is executed outside of inference algorithms:
both the original and the differentiated model are
simultaneously available to the rest of the program code, so
the methods can be called on the differentiated model for
inference, and on the original model for the most efficient
execution with the inferred parameters.

The Go programming language and development environment offer
capabilities which made implementation of \texttt{infergo}
affordable.

\begin{enumerate}
	\item The Go parser and abstract syntax tree serializer are a
		part of the standard library. Reading, transforming, and
		writing Go source code is straightforward and effortless.
	\item Type inference (or \textit{type checking} as it is
		called in the Go ecosystem), also provided in the
		standard library, augments parsing and allows to
		selectively apply transformation-based automatic
		differentiation  based on static expression types. 
	\item Go compiles and runs fast. Fast compilation and
		execution speeds allow to use the same facility for both
		exploratory design of probabilistic models and for
		inference in production environment.
	\item Go offers efficient parallel execution as a
		first-class feature, via so-called \textit{goroutines}.
		Goroutines streamline implementation of sampling-based
		inference algorithms. Sample generators and consumers
		are run in parallel, communicating through channels. 
		Inference is easy to parallelize in order to exploit
		hardware multi-processing, and samples are retrieved
		lazily for postprocessing. 
\end{enumerate}

A lightweight probabilistic programming facility similar to
\texttt{infergo} can be added to most modern general-purpose
programming languages, in particular those used in implementing
large-scale software systems, making probabilistic
programming inference more accessible in server-side
applications.

\bibliography{refs}

\end{document}
