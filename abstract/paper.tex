\documentclass[sigplan,review]{acmart}\settopmatter{printfolios=true,printccs=false,printacmref=false}
\usepackage{url}
\usepackage{hyperref}
\usepackage{listings}[language=Golang]
\usepackage{GoMono}

\lstdefinelanguage{Golang}%
  {morekeywords=[1]{package,import,func,type,struct,return,defer,panic,%
     recover,select,var,const,iota,},%
   morekeywords=[2]{string,uint,uint8,uint16,uint32,uint64,int,int8,int16,%
     int32,int64,bool,float32,float64,complex64,complex128,byte,rune,uintptr,%
     error,interface},%
   morekeywords=[3]{map,slice,make,new,nil,len,cap,copy,close,true,false,%
     delete,append,real,imag,complex,chan,},%
   morekeywords=[4]{for,break,continue,range,go,goto,switch,case,fallthrough,if,%
     else,default,},%
   morekeywords=[5]{Println,Printf,Error,Print,},%
   sensitive=true,%
   morecomment=[l]{//},%
   morecomment=[s]{/*}{*/},%
   morestring=[b]',%
   morestring=[b]",%
   morestring=[s]{`}{`},%
}

\lstset{ % add your own preferences
    xrightmargin=1pt,
    frame=single,
    basicstyle=\ttfamily\footnotesize,
    numbers=none,
    numbersep=4pt,
    showstringspaces=false,
    tabsize=4,
    language=Golang % this is it !
}

\acmConference[LAFI'2020]{Languages for Inference}{January 21,
2020}{New Orleans, LA, USA}

\setcopyright{none}

\bibliographystyle{ACM-Reference-Format}


\title{Probabilistic Programming around Gaussian Processes}
\author{David Tolpin}
\affiliation{
    \institution{PUB+}
    \country{Israel}
}
\email{david.tolpin@gmail.com}

\begin{abstract}
    We present GoGP, a library for probabilistic programming
    around Gaussian processes. Kernels, beliefs about
    hyperparameters and about observation inputs and outputs are
    programmatically expressed using the same differentiable
    probabilistic programming framework. On one hand, a basic
    usage with maximum-likelihood hyperparameter estimates and
    homoscedastic Gaussian noise requires as little coding as
    with any Gaussian process library. On the other hand,
    imposing prior beliefs on hyperparameters, handling input
    uncertainty, heteroscedastic and non-Gaussian noise, or
    change point detection can be organically added on top of a
    Gaussian process kernel. Just like distributions in
    probabilistic programs, kernels can be flexibly combined
    using general code flow rather than through predefined
    combination operators (usually addition and multiplication).
    Implemented in Go, a general-purpose programming language
    popular for server-side programming, our library is both
    well suited for integration into data processing/machine
    learning pipelines and exploits language features such as
	efficient light-weight parallelism for computation
	efficiency.  We demonstrate different uses of the library on
	a variety of case studies inspired by applications. 
\end{abstract}

\begin{document}
\maketitle

\section{Gaussian Processes in Applications}

Gaussian processes~\cite{RW05} possess properties that make
them the approach of choice in many applications:
\begin{itemize}
    \item A Gaussian process works with as little or as much
    	data as available.
    \item Non-uniformly sampled observations, missing
    	observations, and observation noise are handled
    	organically.
    \item Libraries for inference with Gaussian processes
    	are readily available, both standalone and within
    	machine learning frameworks, such that Gaussian
    	processes can be included into data processing
    	pipelines and combined with other machine learning
    	algorithms.
\end{itemize}

Gaussian processes are successfully deployed in many settings,
in particular when 
\begin{itemize}
    \item a stationary kernel is sufficient;
    \item kernel hyperparameters can be fixed \textit{a priori} or
    	reliably estimated from observations;
    \item observation noise is constant.
\end{itemize}
Most basic cases are indeed covered by the above assumptions,
and library implementations~\cite{MWN+17,GPB+18,PVG+11} of
Gaussian processes prioritize scalability with number of
observations, rich libraries of combinable kernel functions, and
integration of Gaussian processes with machine learning
pipelines.

There are cases though where noise varies between
observations, structured informative priors must be imposed on
kernel hyperparameters, observation inputs are uncertain, or
observations are best explained by presence of change
points~\cite{ROE+13,SGR04,GOR09,MR11,STR10}.

While many of the above cases are addressed in research
literature, their diversity is beyond feasible support by a
library, and the practitioner has to implement advanced Gaussian
process algorithms on case-by-case basis. One, laborious, option
is to implement the algorithms in a general-purpose programming
language, along with probabilistic inference for uncertain
quantities --- hyperparameters, observation inputs, location and
number of change points.  This option may, in principle, result
in an efficient and robust implementation, albeit at a high
implementation cost. Another option is to implement the
algorithms in a probabilistic programming
framework~\cite{TMY+16,Stan17,GXG18},
which gains support for probabilistic inference on uncertain
parameters but apparently limits implementation efficiency of
Gaussian process-specific computations.

In this work, we advocate a third approach, which combines
advantages of the other two above. We present an architecture of
a Gaussian process library that provides an efficient
implementation of Gaussian process-related computations shared
by all algorithms, but lets program, virtually without
restrictions, both Gaussian process kernels and probabilistic
models of uncertainty, and combine Gaussian processes and
probabilistic models for inference by means of inference
algorithms supported by the framework. Our implementation of
probabilistic programming-enabled Gaussian processes, GoGP
(\url{http://bitbucket.org/dtolpin/gogp}), is written in Go and
makes extensive use of modelling and inference capabilities
provided by Infergo~\cite{T19}, a Go library for differentiable
probabilistic programming. In what follows we first outline GoGP
design  and then illustrate the use of GoGP on a number of case
studies inspired by applications.

\section{GoGP --- Gaussian Process Library for Programmable Inference}

GoGP is built around a dual view on the Gaussian process: as a
stochastic process and as a probabilistic model with respect
to kernel hyperparameters and, optionally, observations.
\lstinline{GP}, the type encapsulating a Gaussian process,
provides methods \lstinline{Absorb} and \lstinline{Produce} to 
add observations to the process and to predict the distribution
of outputs at given locations. In addition, \lstinline{GP}
provides methods \lstinline{Observe} and \lstinline{Gradient}
that turn it into an Infergo model (with a custom gradient).
Inference can be performed on a \lstinline{GP} instance alone
or on a model that combines a \lstinline{GP} instance with
beliefs about hyperparameters and observations.  The
log-likelihood gradient is computed with respect to
hyperparameters only or with respect to both hyperparameters and
observations, supporting both basic and advanced usage.

Common computationally-intensive algorithms are implemented in 
methods of \lstinline{GP} and rely on efficient parallelism of
the Go language and a highly optimized library for mathematical 
computations Gonum~\cite{Gonum}. Problem-specific code comprises two
parts --- a kernel and optionally a probabilistic model --- that
are both automatically differentiated by Infergo.  As a minimum,
a \lstinline{GP} instance must be initialized with a kernel, an
object of type implementing an Infergo model and computing the
covariance between two locations. A few basic kernels are
provided in GoGP as a convenience, and can be combined in derived
kernels using virtually unrestricted Go code. 

% For example, here is an RBF (Normal) kernel as implemented in
% GoGP (but could also be supplied by the library user):
% \begin{lstlisting}
% type normal struct{}
% 
% func (k normal) Observe(x []float64) float64 {
% 	return k.Cov(x[0], x[1], x[2])
% }
% func (normal) NTheta() int { return 1 }
% func (normal) Cov(l, xa, xb float64) float64 {
% 	d := (xa - xb) / l
% 	return math.Exp(-d * d / 2)
% }
% \end{lstlisting}

\section{Case studies}

GoGP includes case studies, illustrating, on simple examples,
common patterns of GoGP use. We briefly summarize here some
of the case studies.

\paragraph{Basic usage}

In the basic case, similar to that supported by many Gaussian
process libraries, a GP directly serves as the model for
inference on hyperparameters (or the hyperparameters can be just
fixed).

The library user specifies the kernel:
\begin{lstlisting}
type Basic struct{}
func (Basic) Observe(x []float64) float64 {
    return x[0] * kernel.Normal.Observe(x[1:])
}
func (Basic) NTheta() int { return 2 }
\end{lstlisting}
and initializes \lstinline{GP} with a kernel instance:
\begin{lstlisting}
gp := &gp.GP{
    NDim:  1,
    Simil: Basic{},
}
\end{lstlisting}

MLE inference on hyperparameters and prediction can then be performed
through library functions.

\paragraph{Priors on hyperparameters}

If priors on hyperparameters are to be specified, the library
user provides both the kernel and the model.  \lstinline{GP} is
initialized with the kernel, and then \lstinline{GP} and the
model are combined for inference in a derived model:
\begin{lstlisting}
type Model struct {
    gp           *gp.GP
    priors       *Priors
    gGrad, pGrad []float64
}
func (m *Model) Observe(x []float64) float64 {
    var gll, pll float64
    gll, m.gGrad = \
      m.gp.Observe(x), model.Gradient(m.gp)
    pll, m.pGrad = \
      m.priors.Observe(x),model.Gradient(m.priors)
    return gll + pll
}
func (m *Model) Gradient() []float64 {
	for i := range m.pGrad {
		m.gGrad[i] += m.pGrad[i]
	}
	return m.gGrad
}
\end{lstlisting}
In \lstinline{Model}, \lstinline{gp} holds a \lstinline{GP}
instance and \lstinline{priors} holds an instance of the model
expressing beliefs about hyperparameters. A \lstinline{Model}
instance is used for inference on hyperparameters, a
\lstinline{GP} instance --- for prediction.

\paragraph{Uncertain observation inputs}

When observation inputs are uncertain~\cite{MR11}, beliefs about inputs can
be specified, and the log-likelihood gradient can be computed
with respect to both hyperparameters and observation inputs. For
example, in a time series, one can assume that observation
inputs come from a renewal process and let the inputs move 
relative to each other. Then, forecasting can be performed 
relative to posterior observation inputs.

\paragraph{Non-Gaussian noise}

In basic usage, the observation noise is assumed to be Gaussian.
This usage is supported by initializing \lstinline{GP} with a noise
kernel, along with a similarity kernel. When the noise is not
Gaussian~\cite{MWN+17}, an analytical solution for posterior Gaussian process
inference does not always exist. However, non-Gaussian noise is
straightforwardly supported by GoGP through supplying a
model for beliefs on observation outputs:
\begin{lstlisting}
type Noise struct {
	Y []float64 // noisy outputs
}
func (m *Noise) Observe(x []float64) (ll float64){
	// Laplacian noise
	for i := range m.Y {
		ll += Expon.Logp(1/math.Exp(x[0]),
			math.Abs(m.Y[i]-x[1+i]))
	}
	return ll
}
\end{lstlisting}


\bibliography{refs}

\end{document}
